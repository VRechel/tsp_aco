\section{Ausgewählte Algorithmen}{
	Bereits vorgestellt wurden die einzelnen Bestandteile der Architektur, sowie die Architektur als Gesamtbild gezeigt. Im Folgenden sollen beispielhaft die verwendeten Algorithmen thematisch vorgestellt werden. So werden die zentralen Methoden zur Berechnung der Iterationen aus Sicht der Ameisen vorgestellt, sowie auch das Verhalten der Pheromonänderung. Zusätzlich wird die Methode zum Töten einer Ameise beschrieben, welche in so gut wie keiner Implementierung zu finden ist.
	
	\subsection{Ant - iteration()}
	
	
	\subsection{Colony - killAnt()}
	Ebenfalls in dem in Abbildung \ref{uml_class} gezeigten UML-Diagramm erkennbar ist, dass die Ameisenkolonie die Möglichkeit besitzt einzelne Ameisen zu "töten". 
	Diese Methode sollte in einem einwandfreiem Programm keinerlei Verwendung finden, allerdings kann man sich nicht auf eine dauerhafte fehlerfreie Implementierung verlassen. Im Bereich der Software Tests ist dies durch das Prinzip "Fehlen von Fehlern" beschrieben. Dieses sagt aus, dass erfolgreiche Tests nur bestätigen, dass keine Fehler gefunden wurden. Es kann nicht ausgesagt werden, dass keine Fehler vorliegen.\footnote{vgl. \cite{bibid}}
	
	Denn die Methode hat die Funktion, im Falle des Fehlverhaltens eine Ameise aus der Liste der aktiven Ameisen bzw. Threads zu löschen und den Thread zu stoppen.
	Genutzt werden wird diese vor allem im Bereich der aufgefangenen Fehler innerhalb der Implementierung der Ameisen. Sollte eine aufgetretene Exception schwerwiegend und unlösbar sein, wird die Applikation automatisch die Funktion aufrufen.
	
	\subsection{Colony - updatePheromone()}
	
	
}