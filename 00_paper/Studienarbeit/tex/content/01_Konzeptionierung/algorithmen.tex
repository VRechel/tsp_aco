\section{Ausgewählte Algorithmen}{
\label{algorithms}
	Bereits vorgestellt wurden die einzelnen Bestandteile der Architektur, sowie die Architektur als Gesamtbild gezeigt. Im Folgenden sollen beispielhaft die verwendeten Algorithmen thematisch vorgestellt werden. So werden die zentralen Methoden zur Berechnung der Iterationen aus Sicht der Ameisen vorgestellt, sowie auch das Verhalten der Pheromonänderung. Zusätzlich wird die Methode zum Töten einer Ameise beschrieben, welche in so gut wie keiner Implementierung zu finden ist.
	
	\subsection{Ant - iteration()}
	In Kapitel \ref{parameter} wurde bereits die Berechnung beschrieben, die für jede neue Streckenauswahl von den Ameisen durchgeführt werden muss. Diese Berechnung wird in der Implementierung in der Iteration der Ameisen umgesetzt. Pro Durchgang bzw. Iteration wird also jede besuchbare, angrenzende Stadt betrachtet und auf Ihre Attraktivität untersucht. Überwiegt der Pheromonwert $\tau$, welcher über Alpha gewichtet ist, über dem heuristischen Faktor $\eta$, der mit Beta verrechnet wird, so wird die bisher von der Kolonie gewählten Strecke gewählt. Diese Berechnung wird für alle Städte berechnet und dann verglichen. Die im Vergleich attraktivste Strecke wird in diesem Zusammenhang dann gewählt.
	Nachdem eine Strecke gewählt wird, verteilt die Ameise auf der Strecke ihre Pheromone, indem der Kolonie ein Pheromonwert mitgeteilt wird, welcher auf den bisherigen addiert werden muss.
	
	\subsection{Colony - killAnt()}
	Die Kolonie wird eine unübliche Fähigkeit haben, denn sie kann im Problemfall Ameisen töten bzw. vergessen.
	Diese Methode sollte in einem einwandfreiem Programm keinerlei Verwendung finden, allerdings kann man sich nicht auf eine dauerhafte fehlerfreie Implementierung verlassen. Im Bereich der Software Tests ist dies durch das Prinzip "Fehlen von Fehlern" beschrieben. Dieses sagt aus, dass erfolgreiche Tests nur bestätigen, dass keine Fehler gefunden wurden. Es kann nicht ausgesagt werden, dass keine Fehler vorliegen.
	
	Denn die Methode hat die Funktion, im Falle des Fehlverhaltens eine Ameise aus der Liste der aktiven Ameisen bzw. Threads zu löschen. Genutzt werden wird diese vor allem im Bereich der aufgefangenen Fehler innerhalb der Implementierung der Ameisen. Sollte eine aufgetretene Exception schwerwiegend und unlösbar sein, wird die Ameise automatisch die Funktion aufrufen und sich selbst beenden.
	
	\subsection{Colony - updatePheromone()}
	Schon mehrfach erwähnt wurden die Pheromonwerte, die Pheromonmatrix, sowie die Pheromonverteilung. All diese Begriffe sind auf die Ameisenkolonie zurückzuführen. Denn diese enthält die Pheromonmatrix, die von den Ameisen zur Berechnung der Iteration benutzt wird. Um diese aktuell zu halten wird diese von jeder Ameise nach jeder Iteration upgedatet. 
	
	Dabei wird die Funktion \textit{updatePheromone} der Kolonie aufgerufen und ein neuer Wert übergeben. Die Kolonie schreibt diesen dann in die zentrale Matrix, sodass der neue Wert bei der nächsten Iteration allen Ameisen ersichtlich ist. Durch diesen gleichzeitig und unübersichtlichen Zugriff auf die Matrix, muss diese Funktion synchronisiert ablaufen, um einen Datenverlust zu verhindern.
	
}