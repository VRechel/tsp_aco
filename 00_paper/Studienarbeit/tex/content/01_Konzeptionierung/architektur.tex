\section{Architektur}{
	\label{architektur}
	Die Architektur wurde inhaltlich so aufgeteilt, dass eine modulare Implementierung möglich ist. So wurden die folgenden vier Bereiche definiert: Persistenz, Applikation, TSP und ACO
	
	Der Begriff der Persistenz beschreibt in diesem Fall zusätzlich zum dauerhaften Abspeichern der Daten, auch das Einlesen der verschiedenen Problemstellungen. Applikation umfasst den Teil des Programms, der sich nicht direkt mit Daten befasst aber auch nicht zur Problemlösung beiträgt, wie die zwei folgenden Bereiche. Innerhalb des Begriffs TSP werden alle nötigen Parameter und Methoden behandelt, die basierend auf dem \ac{TSP} notwendig werden. Zuletzt gibt es in der Architektur noch das Feld ACO, welches die komplette Berechnung des zu lösenden Problems übernimmt. In dem Fall, dieser Arbeit handelt es sich um das \ac{TSP}. Allerdings könnte das Problemfeld auch dadurch ausgewechselt werden, dass der Bereich \ac{TSP} durch eine  Implementierung des neuen Problems ersetzt wird.
	
	\subsection{Persistenz}
	Um ein starres Definieren des Problems innerhalb der Applikation zu verhindern, werden zu Beginn des Programms alle Parameter, wie Städtematrix, Wahrscheinlichkeiten und Lösungsparameter aus einer gegebenen XML-Datei eingelesen. Durch eine Bearbeitung der XML ist ein einfaches Abändern der Problemstellung möglich. Hierdurch ist auch gesichert, dass die Algorithmen effizient getestet werden können, da mehrere verschiedene bekannte Testwerte benannt werden können.
	
	Ebenfalls Teil der Persistenz ist die Verbindung zur \ac{DB}, welche in diesem Fall eine \ac{HSQLDB} ist. Über diese sollen die Ergebnisse jeder Generation abgespeichert werden. Zusätzlich soll jedes Mal wenn eine bessere Route gefunden wird, diese in einer anderen Tabelle abgespeichert inkl. der Nummer der Generation, welche die Verbesserung bewirkt hat. Dadurch ist es möglich nachzuverfolgen, wann welche Verbesserung eingetreten ist und ab wann der Distanzwert stagniert.
	
	\subsection{Applikation}
	Wie bereits genannt, liegt bei der vorliegenden Architektur ein Fokus auf Modularität, Portabilität und Usability. Aber auch auf verlässliche und effiziente Algorithmen muss geachtet werden. Daher wird für die Wahrscheinlichkeitsberechnung die externe Klasse MersenneTwisterFast \footnote{s. http://www.math.sci.hiroshima-u.ac.jp/~m-mat/MT/emt.html} genutzt, welche eine bessere Distribution der Pseudozufallszahlen bieten als die Default-Implementierung in Java.
	\footnote{Um eine einfache und schnelle Benutzung zu gewährleisten, wird in dieser Arbeit darauf verzichtet echte Zufallszahlen zu nutzen, die beispielsweise aus Weltallstrahlung berechnet werden. Diese seien hier nur zur Vollständigkeit halber erwähnt.}
	Ein weiterer Bestandteil des Applikationsbereichs werden das Logging der Arbeitsvorgänge, sowie die zentrale Konfiguration der Problemstellung auf der die Lösung basiert.
	
	\subsection{TSP}
	Der Bereich des TSP definiert sich in dieser Architektur rein durch die Städte-Objekte, welche zur Darstellung der Städtematrix genutzt werden. Der einzige andere Bestandteil ist die zentrale Aufstellung der Städtematrix, die von den restlichen Klassen referenziert wird. Eine durchdachte Strukturierung der Distanzmatrix zwischen den Städten ist extrem wichtig, da die Performance aller anderen Teile der Applikation auf der Distanzmatrix aufbauen. Ist diese ineffizient oder umständlich aufgebaut, so kann mit dieser auch nicht performant gearbeitet werden.
	
	\subsection{ACO}
	Um den Ameisenalgorithmus umzusetzen sind deutlich mehr Aufwände nötig, als zur Darstellung des TSP. Hier wird mindestens eine Ameisenkolonie benötigt \footnote{Die Architektur würde bei ausreichenden Systemressourcen auch eine parallele Berechnung mehrerer Städtematrizen erlauben}, sowie pro Kolonie mehrere Ameisen. Die Ameisen werden in der Applikation als einzelne Threads gestartet, die über eine CyclicBarrier kontrolliert werden. Dies erlaubt ein möglichst effizientes Parallelisieren der Lösung.
	
	In dem Bereich \ac{ACO} gibt es im Konzept mehrere Besonderheiten, welche zu einem besseren Ergebnis führen sollen. Vorhanden sind diese vor allem im Bereich der Kolonie, sowie der Ameisen. Der Kolonie soll es möglich sein, falls eine bessere Route gefunden wird als die bisherige, die neue Route doppelt in der Pheromonmatrix zu gewichten. Die Kolonie belegt also die Strecke zusätzlich zu den Pheromonen der Ameise nochmals mit neuen Pheromonen.
	
	Die Ameise hat zum normalen Vorgehen noch einen Filter implementiert, welcher dafür sorgen soll, dass die Wegsuche nicht ab einem Zeitpunkt komplett stagniert und nur noch die gleiche Route befolgt wird. Dieser ``Idiocrazy''\footnote{Benannt nach dem Film ``Idiocrazy'' aus dem Jahr 2006. In der Satire-Komödie wird eine Welt im Jahr 2500 behandelt, die komplett verblödet ist.}-Filter sorgt dafür, dass eine Ameise vor der Berechnung der Wahrscheinlichkeiten, später im Kapitel \ref{parameter} behandelt, zuerst eine Zufallszahl berechnen lässt. Sollte diese Zufallszahl kleiner als ein definierter Wert\footnote{Derzeitiger Standard: 0.5 \%} sein, so bestimmt die Ameise eine zufällige Stadt aus der Menge der erreichbaren aus und geht zu dieser. Sollte der Filter in Kraft treten, werden für den derzeitigen Schritt keinerlei Berechnungen außer den Zufallszahlen durchgeführt. Auch werden die Distanz- und Pheromonwerte ignoriert, wodurch neue Routen erschlossen werden können. In Verbindung mit einer höher gewichteten besten Route kann dieses Verfahren zu einer besseren und schnelleren Berechnung der Routen führen.
	
	\subsection{Arbeitsweise der Architektur}
\label{numerschiesBeispiel}
	Die einzelnen Abschnitte der Architektur wurden bereits erklärt. Aber das Zusammenspiel der einzelnen Komponenten und der eigentliche Arbeitsablauf des Systems wurde noch nicht beschrieben. Im Folgenden wird ein Beispiel so durchgeführt, wie es auch die geplante Architektur umsetzen würde.
	\begin{figure}[H]
		\centering
		\includegraphics[width=0.5\linewidth]{images/TSP_ACO_numerisch.png}
		\caption{Darstellung des \ac{TSP}-Beispiels zur Darlegung der Arbeitsweise der Architektur}
		\label{tspAcoNumerisch}
	\end{figure}
	In Abbildung \ref{tspAcoNumerisch} ist ein Beispiel für das \ac{TSP} gegeben. Sechs Städte sind untereinander so vernetzt, dass jede Stadt von jeder anderen erreichbar ist. Auch ist Stadt A rot markiert, wodurch diese als Standort für die Ameisenkolonie ausgewählt wurde. Nun wird die gleiche Berechnung aufgezeigt, welche die Applikation durchführen würde.

	In Tabelle \ref{tspAcoNumerisch_matrix} zu sehen ist die allgemeingültige Streckenmatrix für das vorliegende Beispiel. Innerhalb der Matrix sind jeweils die Streckenlängen von einer Stadt zur Anderen gespeichert. So besitzt die Strecke von D nach C die Länge 4\footnote{Hier könnte natürlich eine beliebige Längeneinheit ergänzt werden. Dies ist aber für das Lösen des Problems und das Beschreiben des Vorgehens nicht notwendig.}. Auffällig sind die Strecken von den Städten zu sich selbst, welche mit -1 gekennzeichnet sind. Diese Zahl wird später für die Applikation ein Hinweis sein, dass ein Fehler vorliegt und dieser korrigiert werden muss.
	\begin{table}[h]
		\centering
		\footnotesize
		\setstretch{0.75}
		\begin{tabular}{c c c c c c c}
			  & A & B & C & D & E & F \\
			A & -1 & 3 & 9 & 13 & 11 & 5\\ 
			B & 3 & -1 & 7 & 10 & 9 & 8\\ 
			C & 9 & 7 & -1 & 4 & 6 & 10\\
			D & 13 & 10 & 4 & -1 & 2 & 12\\
			E & 11 & 9 & 6 & 2 & -1 & 8\\
			F & 5 & 8 & 10 & 12 & 8 & -1\\
		\end{tabular}
		\caption{Distanzmatrix des numerischen \ac{TSP}-Beispiels}
		\label{tspAcoNumerisch_matrix}
	\end{table}
	Besonders wichtig für die Berechnung der \ac{ACO}-Lösung ist die Pheromonmatrix, welche in Tabelle \ref{tspAcoNumerisch_pheromon_initial} im initialisierten Zustand gezeigt ist. Die Normalisierung mit einem durchgängigen Wert von 1 bewirkt, dass im ersten Durchlauf die Pheromone keinerlei Einfluss auf die Entscheidungen der Ameisen haben. Dies entspricht dem Umstand einer neu aufgebauten Ameisenkolonie, welche sich erst zurecht finden muss.	
	\begin{table}[h]
		\centering
		\footnotesize
		\setstretch{0.75}
		\begin{tabular}{c c c c c c c}
			& A & B & C & D & E & F \\
			A & 1 & 1 & 1 & 1 & 1 & 1\\ 
			B & 1 & 1 & 1 & 1 & 1 & 1\\ 
			C & 1 & 1 & 1 & 1 & 1 & 1\\
			D & 1 & 1 & 1 & 1 & 1 & 1\\
			E & 1 & 1 & 1 & 1 & 1 & 1\\
			F & 1 & 1 & 1 & 1 & 1 & 1\\
		\end{tabular}
		\caption{Initiale Pheromonmatrix des \ac{TSP}-Beispiels}
		\label{tspAcoNumerisch_pheromon_initial}
	\end{table}
	Mit den vorliegenden Daten, der Streckenlängenmatrix und der initialen Matrix, kann nun das \ac{TSP} mithilfe von \ac{ACO} berechnet werden. In diesem Beispiel werden drei Ameisen verwendet, um das Verhalten zeigen zu können, aber gleichzeitig den Aufwand gering zu halten.
	
	Jede Ameise startet bei der Kolonie, welche bei Stadt A liegt. Daher muss für jede Ameise jetzt ausgewählt werden, wohin diese gehen soll. Für jede mögliche Strecke wird nun eine Variabel berechnet, welche im Folgenden mit $\lambda$ \footnote{Mehr zu diesem Wert in Kapitel \ref{parameter}.} bezeichnet wird. $\lambda$ setzt sich zusammen aus dem Pheromonwert der betrachteten Strecke und der zugehörigen Streckenlänge.

	Nun wird für jede Ameise einzeln betrachtet, welche Städte noch erreichbar sind - also noch nicht besucht sind - und welchen Wert $\lambda$ für die Strecke zu dieser Stadt besitzt. $\lambda$ einer Stadt wird dann mit der Summe aus den $\lambda$ aller erreichbaren Städten verglichen. Hierdurch wird ein Wert - im Folgenden mit $\rho$\footnote{Auch zu diesem Wert mehr in Kapitel \ref{parameter}.} referenziert - erzeugt, welcher zwischen 0 und 1 liegt. 
	
	Um nun bestimmen zu können, welche Ameise welchen Weg wählt, wird eine Zufallszahl $n$ bestimmt, welche ebenfalls zwischen 0 und 1 liegt. Nacheinander wird für alle erreichbaren Städte nun verglichen, ob sich $\rho>n$ ergibt. Sobald eine Stadt die Gleichung erfüllt, wählt die Ameise diesen Weg und zieht weiter. Erfüllt keine Stadt diese Gleichung, so wird nacheinander  $\rho$ der Städte aufaddiert, bis eine Stadt x die Gleichung $n < \sum_x^{} \rho$ erfüllt. Wenn diese Gleichung erfüllt ist, wählt die Ameise der Strecke zu Stadt x.
	
	In dem vorliegenden Beispiel bedeutet das, dass von Stadt A aus alle möglichen Strecken ausgewertet werden. Folgend ist die Berechnung von $\lambda$ aller erreichbaren Städte aufgezeigt.
	\begin{subequations}
		\begin{equation}
			\lambda = \tau _{ij}^{\alpha } * \eta _{ij}^{\beta}
		\end{equation}
		\begin{align}
			A -> B \quad (1)^2 * (\frac{1}{3}) = \frac{1}{3}\\
			A -> C \quad (1)^2 * (\frac{1}{5}) = \frac{1}{5}\\
			A -> D \quad (1)^2 * (\frac{1}{9}) = \frac{1}{9}\\
			A -> E \quad (1)^2 * (\frac{1}{13}) = \frac{1}{13}\\
			A -> F \quad (1)^2 * (\frac{1}{11}) = \frac{1}{11}
		\end{align}
		\myequations{Formeln zum Berechnen des Verhältnisses $\lambda$ von Pheromonwert und Streckenlänge}
	\end{subequations}

	Aus diesen kann nun $\rho$ errechnet werden, in dem alle $\lambda$ durch die Summe aller $\lambda$ - welche 0,812 beträgt - dividiert werden. $\rho$ steht dann für die Wahrscheinlichkeit, dass dieser Weg gewählt wird. Somit ergeben sich folgende fünf Wahrscheinlichkeiten:
	\begin{subequations}
		\begin{equation}\label{eq:P}
			P(s_{ij}) = \frac{\lambda}{\sum_{}^{ } \lambda}
		\end{equation}
		\begin{align}
			\rho(AB) = \frac{\frac{1}{3}}{0,8122} = 0.410	\\
			\rho(AC) = \frac{\frac{1}{5}}{0,8122} = 0.136	\\
			\rho(AD) = \frac{\frac{1}{9}}{0,8122} = 0.094	\\
			\rho(AE) = \frac{\frac{1}{13}}{0,8122} = 0.111	\\
			\rho(AF) = \frac{\frac{1}{11}}{0,8122} = 0.246
		\end{align}
		\myequations{Formeln zum Berechnen der Wahrscheinlichkeiten $\rho$ aus dem Verhältnis $\lambda$ und der Summe aller $\lambda$}
	\end{subequations}	
	Nun muss für jede der drei Ameisen eine Zufallszahl bestimmt werden, welche mit $\rho$ verglichen werden kann. Es wurden die Zahlen 0.242 für Ameise 1, 0.033 für Ameise 2 und 0.455 für Ameise 3 errechnet.
	Nach dem bereits beschriebenen Vorgehen wählen Ameise 1 und Ameise 2 nun die Stadt B als Zielstadt. Ameise 3 wählt Stadt C, da der Zufallswert zu groß war um eine direkte Auswahl zu treffen. Ein Aufaddieren der Werte $\rho(AB)$ und $\rho(AC)$ hat allerdings schon ausgereicht, um den Wert zu überschreiten.
	
	Diesen Vorgehen kann nun für alle Städte wiederholt werden, sodass am Ende alle Ameisen alle Städte besucht haben und auch wieder in der Anfangsstadt in der Kolonie angekommen sind. Nun muss noch die Pheromonmatrix aktualisiert werden, um der Kolonie mitzuteilen, welche Wege profitable sind. Hierbei wird von jeder Ameise auf jedem Weg den sie abgelaufen ist Pheromone abgegeben. Es ergibt sich als zusätzlicher Pheromonwert der Kehrwert der Streckenlänge. Der Kehrwert wird auf den aktuellen Wert in der Pheromonmatrix addiert, wodurch die nächste Generation diese in die Berechnung einbeziehen kann.
	
	\newpage
	In Tabelle \ref{tspAcoNumerisch_ergebnis_1} zu erkennen sind die kompletten Wegstrecken, die die Ameisen gewählt haben. Aus dieser Liste ableiten lassen sich nun die zusätzlichen Pheromonwerte, in dem man die Tabelle \ref{tspAcoNumerisch_matrix} miteinbezieht. Addiert man die zusätzlichen Pheromonwerte auf die vorherige Pheromonmatrix, so erhält man die in Tabelle \ref{tspAcoNumerisch_pheromon_1} gezeigte neue Pheromonmatrix.
	\begin{table}[h]
		\centering
		\footnotesize
		\setstretch{0.75}
		\begin{tabular}{l c r}
			Ameise & Weglänge & Wegstrecke \\
			1 & 40 & A, B, F, D, E, C, A\\
			2 & 36 & A, B, F, E, D, C, A\\ 
			3 & 50 & A, C, D, B, F, E, A\\
		\end{tabular}
		\caption{Ergebnisse der drei Ameisen der ersten Generation}
		\label{tspAcoNumerisch_ergebnis_1}
	\end{table}
	\begin{table}[h]
		\centering
		\footnotesize
		\setstretch{0.75}
		\begin{tabular}{c c c c c c c}
			& A & B & C & D & E & F \\
			A & 1 & $\frac{5}{3}$ & $\frac{10}{9}$ & 1 & 1 & 1\\ 
			B & 1 & 1 & 1 & 1 & 1 & $\frac{5}{4}$\\ 
			C & $\frac{11}{9}$ & 1 & 1 & $\frac{5}{4}$ & 1 & 1\\
			D & 1 & $\frac{11}{10}$ & $\frac{5}{4}$ & 1 & $\frac{3}{2}$ & 1\\
			E & $\frac{12}{11}$ & 1 & $\frac{7}{6}$ & $\frac{3}{2}$ & 1 & 1\\
			F & 1 & 1 & 1 & $\frac{13}{12}$ & $\frac{5}{4}$ & 1\\
		\end{tabular}
		\caption{Pheromonmatrix des TSP-Beispiels nach dem ersten Durchlauf}
	\label{tspAcoNumerisch_pheromon_1}
	\end{table}
	Nach der Berechnung der neuen Pheromonmatrix ist die erste Generation der Ameisen abgeschlossen. Nun werden sich drei neue Ameisen auf die Reise begeben. Da die Pheromonmatrix nicht mehr normalisiert ist, sondern von 1 abweichende Werte enthält, können die neuen Ameisen die Pheromonmatrix effektiv in die Berechnung der Wahrscheinlichkeiten miteinbeziehen. So wird beispielsweise $\lambda$ für die Strecke A - B nun mit folgender Gleichung berechnet:	
	\begin{equation}
		A -> B \quad (\frac{5}{3})^2 * (\frac{1}{3})
	\end{equation}
	Hierbei wird nun das Verhältnis zwischen Pheromonen und Streckenlänge betrachtet, wobei die Pheromone doppelt so schwer gewichtet werden. Die restliche Berechnung läuft allerdings parallel zum vorherigen Vorgehen ab. Die zweite Generation der Ameisen hat bei einigen Strecken eine andere Wahl getroffen, wie man aus dem Vergleich zwischen Tabelle \ref{tspAcoNumerisch_ergebnis_1} und Tabelle \ref{tspAcoNumerisch_ergebnis_2} erkennt. Im Durchschnitt ist die neue Generation auch schneller voran gekommen, was an den Weglängen - also der Summe aller gelaufenen Strecken - erkennbar ist.
	\newpage
	Auch diese Generation verteilt auf allen besuchten Strecken ihre Pheromone, was wieder zu einer Aktualisierung der Pheromonmatrix führt. In Tabelle \ref{tspAcoNumerisch_pheromon_2} zu erkennen ist, dass nun deutlich mehr Streckenabschnitte besucht wurden und einen von 1 abweichenden Pheromonwert besitzen.
	\begin{table}[H]
		\centering
		\footnotesize
		\setstretch{0.75}
		\begin{tabular}{l c r}
			Ameise & Weglänge & Wegstrecke \\
			1 & 40 & A, C, E, D, B, F, A\\
			2 & 29 & A, F, E, D, C, B, A\\ 
			3 & 41 & A, F, E, D, B, C, A\\
		\end{tabular}
		\caption{Ergebnisse der drei Ameisen der zweiten Generation}
		\label{tspAcoNumerisch_ergebnis_2}
	\end{table}
	\begin{table}[H]
		\centering
		\footnotesize
		\setstretch{0.75}
		\begin{tabular}{c c c c c c c}
			& A & B & C & D & E & F \\
			A & 1 & $\frac{5}{3}$ & $\frac{11}{9}$ & 1 & 1 & $\frac{7}{5}$\\ 
			B & $\frac{4}{3}$ & 1 & $\frac{8}{7}$ & 1 & 1 & $\frac{11}{8}$\\ 
			C & $\frac{4}{3}$ & $\frac{11}{10}$ & 1 & $\frac{5}{4}$ & $\frac{7}{6}$ & 1\\
			D & 1 & $\frac{13}{10}$ & $\frac{3}{2}$ & 1 & $\frac{3}{2}$ & 1\\
			E & $\frac{12}{11}$ & 1 & $\frac{7}{6}$ & 3 & 1 & 1\\
			F & $\frac{6}{5}$ & 1 & 1 & $\frac{13}{12}$ & $\frac{3}{2}$ & 1\\
		\end{tabular}
		\caption{Pheromonmatrix des \ac{TSP}-Beispiels nach dem zweiten Durchlauf}
		\label{tspAcoNumerisch_pheromon_2}
	\end{table}
	Nach der zweiten Generation macht sich noch eine letzte Generation der Ameisen auf den Weg und läuft wieder die gleichen Orte ab. Wieder wird die vorher aktualisierte Pheromonmatrix in die Gewichtung miteinbezogen, um ein effizienteres Ergebnis zu erhalten. Betrachtet man das in Tabelle \ref{tspAcoNumerisch_ergebnis_3} gezeigte Ergebnis der Ameisen, so ist deutlich sichtbar dass eine Verbesserung vorliegt.
	\begin{table}[h]
		\centering
		\footnotesize
		\setstretch{0.75}
		\begin{tabular}{c c c c c c c}
			& A & B & C & D & E & F \\
			A & 1 & $\frac{8}{3}$ & $\frac{11}{9}$ & 1 & 1 & $\frac{7}{5}$\\ 
			B & $\frac{4}{3}$ & 1 & $\frac{10}{7}$ & 1 & $\frac{10}{9}$ & $\frac{11}{8}$\\ 
			C & $\frac{4}{3}$ & $\frac{11}{10}$ & 1 & $\frac{3}{2}$ & $\frac{8}{6}$ & $\frac{11}{10}$\\
			D & 1 & $\frac{13}{10}$ & $\frac{7}{4}$ & 1 & 2 & $\frac{13}{12}$\\
			E & $\frac{12}{11}$ & 1 & $\frac{7}{6}$ & 4 & 1 & $\frac{9}{8}$\\
			F & $\frac{9}{5}$ & 1 & 1 & $\frac{13}{12}$ & $\frac{3}{2}$ & 1\\
		\end{tabular}
		\caption{Pheromonmatrix des \ac{TSP}-Beispiels nach dem finalen dritten Durchlauf}
		\label{tspAcoNumerisch_pheromon_3}
	\end{table}
	Abschließend lässt sich aus der finale Pheromonmatrix - in Tabelle \ref{tspAcoNumerisch_pheromon_3} gezeigt - erkennen, dass zum Einen ein Großteil der Strecken besucht wurde. Zum Anderen liegt bereits nach drei Generationen teilweise eine deutliche Gewichtung vor, sodass bei Stadt E meist nur noch die Stadt D als Ziel gewählt wird - falls diese noch erreichbar ist.
	Es lässt sich bereits bei dieser manuellen Berechnung schlussfolgern, dass mehr Ameisen wahrscheinlich nur bedeuten, dass die Pheromonverteilung schneller angepasst und optimiert wird. Dies hat zur Folge, dass bei höherer Ameisenzahl weniger Generationen benötigt werden, um ein besseres Ergebnis zu erhalten. Allerdings hat es keinen Einfluss auf das Endergebnis.
	\begin{table}[h]
		\centering
		\footnotesize
		\setstretch{0.75}
		\begin{tabular}{l c r}
			Ameise & Weglänge & Wegstrecke \\
			1 & 35 & A, B, C, E, D, F, A\\
			2 & 33 & A, B, E, D, C, F, A\\ 
			3 & 29 & A, B, C, D, E, F, A\\
		\end{tabular}
		\caption{Ergebnisse der drei Ameisen der dritten Generation}
		\label{tspAcoNumerisch_ergebnis_3}
	\end{table}
}