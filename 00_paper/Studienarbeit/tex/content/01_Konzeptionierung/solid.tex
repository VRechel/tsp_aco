\section{Umsetzung der SOLID-Prinzipien}{
	Um eine saubere und übersichtliche Implementierung gibt es heutzutage eine Vielzahl von Regelwerken, Anleitungen und Vorgaben. Eine Sammlung von Grundsätzen sind die SOLID-Prinzipien. SOLID steht für \textbf{S}ingle Responsibility, \textbf{O}pen/Closed, \textbf{L}iskov Substitution, \textbf{I}nterface Segregation und \textbf{D}ependency Inversion.
	Alle diese Eigenschaften werden im Folgenden kurz erklärt und dann ihre Verwirklichung in der Architektur gezeigt.
	
	\subsection{Single Responsibility}
	Das Single-Responsibility-Prinip umschreibt den Zustand, dass Klassen max. eine Zuständig haben sollen. Klassen dürfen nicht mehrere Aufgabengebiete gleichzeitig übernehmen, da hierdurch eine Änderung an dieser Klasse zu einer Änderung mehrerer Komponenten führt und dadurch die ganze Struktur beeinflusst werden kann.
	Stattdessen wird die Software in mehrere kleinere Klassen aufgeteilt, die jeweils einen Teil übernehmen. Hierdurch ist zum Einen die Struktur einfacherer erkennbar und nachvollziehbar, da die Klassen eindeutiger definiert sind und der Code besser zusammengefasst ist. Zum Anderen sind die Abhängigkeiten aber auch auf ein Minimum reduziert.
	
	In der vorliegenden Konzeption ist dieser Grundsatz dadurch erfüllt, dass die Berechnungslogik an die Ameisen ausgegliedert ist, welche selbst keine Manipulation vornehmen. Die Ameisenkolonie hingegen funktioniert nur als Quelle der Threads und behält einige wenige Kolonie-spezifische Attribute bereit, sonst nichts.
	Andere Anwendungen, wie das Logging, das Einlesen von Konfigurationsdateien und die Zufallszahlerzeugung wurde alle in getrennte Klassen ausgegliedert, die zentral in \textit{Configuration} initialisiert werden.
	
	\subsection{Open / Closed}
	Die Open-Closed-Eigenschaft, welche eine moderen Software besitzen muss, umschreibt den Umstand, dass ein möglichst modularer Aufbau genutzt wird. So sollen Klassen so aufgebaut sein, dass bei einer Erweiterung keine Änderung bestehenden Codes notwendig wird, sondern zum Beispiel über das Implementieren eines Interfaces die alten Strukturen genutzt werden können.
	Unterscheiden muss man hier zwischen einer Erweiterung und einer Änderung. Die Software muss nicht und soll auch nicht auf eine einfache Änderung ausgelegt sein. Der Code, welcher erfolgreich implementiert wurde, soll auch in seinem Funktionsumfang weiter genutzt werden. Die bestehenden Implementierungen müssen keine Schnittstellen zur einfach Änderung enthalten.
	
	Durch die Schlichtheit der entworfenen Architektur und den auf die Problemstellung zugeschnittenen Charakter ist eine Umsetzung des Open-Closed-Prinzips nur in geringem Umfang möglich. Alleine die Parser können hiernach umgesetzt werden, in dem ein Interface eingesetzt wird. Dadurch wird ermöglich in Zukunft auch andere Datei-Typen akzeptieren zu können.
	In der restlichen Implementierung ist eine Umsetzung nicht hilfreich oder zweckführend.
	
	\subsection{Liskov Substitution}
	Die Substitutionsregel von Liskov besagt, dass es möglich sein sollte eine Klasse in einem beliebigen Aufruf auch durch eine Subklasse aus zu tauschen ohne den Programmablauf zu ändern.
	Folglich müssen entweder die Implementierung abgestimmt sein, dass ein Austauschen generell möglich ist. Dies ist aber meist nicht zweckdienlich.
	Andererseits ist es auch möglich, über eine abstrakte Klasse zu arbeiten, die für Methodenaufrufe benutzt wird. So wird der Programmablauf nicht durch unterschliede Klassentypen unterbrochen, sondern der Entwickler ist in der Pflicht die abstrakte Klasse entsprechend umzusetzen.
	
	Ähnlich zu der Open-Closed-Eigenschaft ist auch diese Leitlinie in dem vorliegenden Entwurf nur sehr schwierig umzusetzen. Da von ein Hineinquetschen von Regeln und Leitlinien in eine Architektur generell abzuraten ist, wurde auf eine Umsetzung der Substitutionsregel generell verzichtet.
	
	\subsection{Interface Segregation}
	Die Trennung der Interfaces zielt darauf ab, Klassen nicht dazu zu zwingen Methoden zu implementieren, die gar nicht benötigt werden. So müssen in den meisten Programmiersprachen alle Methoden eines Interfaces zwingend umgesetzt werden. Dies führt aber meist zu einer aufgeblähten Codestruktur, da unnötige Methoden implementiert werden müssen, aber nie benutzt werden.
	Indem man die Interfaces in kleine Interfaces unterteilt ist es möglich durch eine Implementierung von mehrerer Interfaces auf das gleiche Ergebnis zu kommen ohne den Zwang alle anderen Interface auch umzusetzen.
	
	Aufgrung des Mangels an Interfaces in der Codestruktur der hier behandelten Architektur ist auch diese Regel nicht umgesetzt. Sobald Interface allerdings zum Einsatz kommen, muss diese Regel umgesetzt werden.
	
	\subsection{Dependency Inversion}
	Die Abhängigkeitsumkehr-Regel besagt, dass Module auf höheren Ebenen nicht auf Module niederer Ebenen angewiesen sein dürfen. Ähnlich zu den anderen Richtlinien sollen auch hier abstrakte Klassen und Interface zur Abstraktion genutzt werden.
	Allerdings gibt es noch den Zusatz, dass Abstraktionen niemals von einer detaillierten Implementierung abhängen dürfen.

	Wie bei den anderen Interface-Umsetzungen ist auch hier eine Umsetzung nicht hilfreich bzw. möglich. Dennoch sei auch hier erwähnt, dass eine Benutzung der Regel zu wartbarerem Code führt.
}